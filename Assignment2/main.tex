\documentclass{report}

\usepackage[utf8]{inputenc}
\usepackage[T1]{fontenc}
\usepackage{textcomp}
\usepackage[english]{babel}
\usepackage{amsmath, amssymb}
\usepackage{enumitem}
\usepackage{bm}
% figure support
\usepackage{import}
\usepackage{xifthen}
\pdfminorversion=7
\usepackage{pdfpages}
\usepackage{transparent}

% \newcommand{\incfig}[1]{%
% 	\def\svgwidth{\columnwidth}
% 	\import{./figures/}{#1.pdf_tex}
% }
\pdfsuppresswarningpagegroup=1

\usepackage[tmargin=2cm,rmargin=1in,lmargin=1in,margin=0.85in,bmargin=2cm,footskip=.2in]{geometry}
\usepackage{amsmath,amsfonts,amsthm,amssymb,mathtools}
\usepackage[varbb]{newpxmath}
\usepackage{xfrac}
\usepackage[makeroom]{cancel}
\usepackage{mathtools}
\usepackage{bookmark}
\usepackage{enumitem}
\usepackage{hyperref,theoremref}
\hypersetup{
	pdftitle={assignment},
	colorlinks=true, linkcolor=doc!90,
	bookmarksnumbered=true,
	bookmarksopen=true
}
\usepackage[most,many,breakable]{tcolorbox}
\usepackage{xcolor}
\usepackage{varwidth}
\usepackage{varwidth}
\usepackage{etoolbox}
%\usepackage{authblk}
\usepackage{nameref}
\usepackage{multicol,array}
\usepackage[ruled,vlined,linesnumbered]{algorithm2e}
\usepackage{comment} % enables the use of multi-line comments (\ifx \fi) 
\usepackage{import}
\usepackage{xifthen}
\usepackage{pdfpages}
\usepackage{transparent}
\usepackage{chngcntr}
\usepackage{tikz}
\usepackage{titletoc}

\newcommand\myfunfont[1]{\ttfamily\textcolor{blue}{#1}}
\SetCommentSty{mycommfont}
\newcommand{\incfig}[1]{%
	\def\svgwidth{\columnwidth}
	\import{./figures/}{#1.pdf_tex}
}

\SetStartEndCondition{ }{}{}%
\SetKwProg{Fn}{\myfunfont{function}}{\string:}{}
\SetKwFunction{Range}{range}%%
\SetKw{KwTo}{in}\SetKwFor{For}{\myfunfont{for}}{\string:}{}%
\SetKwIF{If}{ElseIf}{Else}{\myfunfont{if}}{:}{\myfunfont{elif}}{\myfunfont{else}:}{}%
\SetKwFor{While}{\myfunfont{while}}{:}{\myfunfont{end}}%
\SetKw{Return}{\myfunfont{return}}
\newcommand{\forcond}{$i=0$ \KwTo $n$}
\renewcommand{\forcond}{$i$ \KwTo\Range{$n$}}
\AlgoDontDisplayBlockMarkers\SetAlgoNoEnd\SetAlgoNoLine%
\usepackage{tikzsymbols}
\tikzset{
	symbol/.style={
			draw=none,
			every to/.append style={
					edge node={node [sloped, allow upside down, auto=false]{$#1$}}}
		}
}
\tikzstyle{c} = [circle,fill=black,scale=0.5]
\tikzstyle{b} = [draw, thick, black, -]
\tikzset{
	vertex/.style={
			circle,
			draw,
			minimum size=6mm,
			inner sep=0pt
		}
}
\renewcommand\qedsymbol{$\Laughey$}

%\usepackage{import}
%\usepackage{xifthen}
%\usepackage{pdfpages}
%\usepackage{transparent}


%%%%%%%%%%%%%%%%%%%%%%%%%%%%%%
% SELF MADE COLORS
%%%%%%%%%%%%%%%%%%%%%%%%%%%%%%

\definecolor{doc}{RGB}{0,60,110}
\definecolor{myg}{RGB}{56, 140, 70}
\definecolor{myb}{RGB}{45, 111, 177}
\definecolor{myr}{RGB}{199, 68, 64}
\definecolor{mytheorembg}{HTML}{F2F2F9}
\definecolor{mytheoremfr}{HTML}{00007B}
\definecolor{mylemmabg}{HTML}{FFFAF8}
\definecolor{mylemmafr}{HTML}{983b0f}
\definecolor{mypropbg}{HTML}{f2fbfc}
\definecolor{mypropfr}{HTML}{191971}
\definecolor{myexamplebg}{HTML}{F2FBF8}
\definecolor{myexamplefr}{HTML}{88D6D1}
\definecolor{myexampleti}{HTML}{2A7F7F}
\definecolor{mydefinitbg}{HTML}{E5E5FF}
\definecolor{mydefinitfr}{HTML}{3F3FA3}
\definecolor{notesgreen}{RGB}{0,162,0}
\definecolor{myp}{RGB}{197, 92, 212}
\definecolor{mygr}{HTML}{2C3338}
\definecolor{myred}{RGB}{127,0,0}
\definecolor{myyellow}{RGB}{169,121,69}
\definecolor{myexercisebg}{HTML}{F2FBF8}
\definecolor{myexercisefg}{HTML}{88D6D1}

%%%%%%%%%%%%%%%%%%%%%%%%%%%%
% TCOLORBOX SETUPS
%%%%%%%%%%%%%%%%%%%%%%%%%%%%

\setlength{\parindent}{1cm}
%================================
% THEOREM BOX
%================================

\tcbuselibrary{theorems,skins,hooks}
\newtcbtheorem[number within=section]{Theorem}{Theorem}
{%
	enhanced,
	breakable,
	colback = mytheorembg,
	frame hidden,
	boxrule = 0sp,
	borderline west = {2pt}{0pt}{mytheoremfr},
	sharp corners,
	detach title,
	before upper = \tcbtitle\par\smallskip,
	coltitle = mytheoremfr,
	fonttitle = \bfseries\sffamily,
	description font = \mdseries,
	separator sign none,
	segmentation style={solid, mytheoremfr},
}
{th}

\tcbuselibrary{theorems,skins,hooks}
\newtcbtheorem[number within=chapter]{theorem}{Theorem}
{%
	enhanced,
	breakable,
	colback = mytheorembg,
	frame hidden,
	boxrule = 0sp,
	borderline west = {2pt}{0pt}{mytheoremfr},
	sharp corners,
	detach title,
	before upper = \tcbtitle\par\smallskip,
	coltitle = mytheoremfr,
	fonttitle = \bfseries\sffamily,
	description font = \mdseries,
	separator sign none,
	segmentation style={solid, mytheoremfr},
}
{th}


\tcbuselibrary{theorems,skins,hooks}
\newtcolorbox{Theoremcon}
{%
	enhanced
	,breakable
	,colback = mytheorembg
	,frame hidden
	,boxrule = 0sp
	,borderline west = {2pt}{0pt}{mytheoremfr}
	,sharp corners
	,description font = \mdseries
	,separator sign none
}

%================================
% Corollery
%================================
\tcbuselibrary{theorems,skins,hooks}
\newtcbtheorem[number within=section]{Corollary}{Corollary}
{%
	enhanced
	,breakable
	,colback = myp!10
	,frame hidden
	,boxrule = 0sp
	,borderline west = {2pt}{0pt}{myp!85!black}
	,sharp corners
	,detach title
	,before upper = \tcbtitle\par\smallskip
	,coltitle = myp!85!black
	,fonttitle = \bfseries\sffamily
	,description font = \mdseries
	,separator sign none
	,segmentation style={solid, myp!85!black}
}
{th}
\tcbuselibrary{theorems,skins,hooks}
\newtcbtheorem[number within=chapter]{corollary}{Corollary}
{%
	enhanced
	,breakable
	,colback = myp!10
	,frame hidden
	,boxrule = 0sp
	,borderline west = {2pt}{0pt}{myp!85!black}
	,sharp corners
	,detach title
	,before upper = \tcbtitle\par\smallskip
	,coltitle = myp!85!black
	,fonttitle = \bfseries\sffamily
	,description font = \mdseries
	,separator sign none
	,segmentation style={solid, myp!85!black}
}
{th}


%================================
% LEMMA
%================================

\tcbuselibrary{theorems,skins,hooks}
\newtcbtheorem[number within=section]{Lemma}{Lemma}
{%
	enhanced,
	breakable,
	colback = mylemmabg,
	frame hidden,
	boxrule = 0sp,
	borderline west = {2pt}{0pt}{mylemmafr},
	sharp corners,
	detach title,
	before upper = \tcbtitle\par\smallskip,
	coltitle = mylemmafr,
	fonttitle = \bfseries\sffamily,
	description font = \mdseries,
	separator sign none,
	segmentation style={solid, mylemmafr},
}
{th}

\tcbuselibrary{theorems,skins,hooks}
\newtcbtheorem[number within=chapter]{lemma}{lemma}
{%
	enhanced,
	breakable,
	colback = mylemmabg,
	frame hidden,
	boxrule = 0sp,
	borderline west = {2pt}{0pt}{mylemmafr},
	sharp corners,
	detach title,
	before upper = \tcbtitle\par\smallskip,
	coltitle = mylemmafr,
	fonttitle = \bfseries\sffamily,
	description font = \mdseries,
	separator sign none,
	segmentation style={solid, mylemmafr},
}
{th}

%================================
% Exercise
%================================

\tcbuselibrary{theorems,skins,hooks}
\newtcbtheorem[number within=section]{Exercise}{Exercise}
{%
	enhanced,
	breakable,
	colback = myexercisebg,
	frame hidden,
	boxrule = 0sp,
	borderline west = {2pt}{0pt}{myexercisefg},
	sharp corners,
	detach title,
	before upper = \tcbtitle\par\smallskip,
	coltitle = myexercisefg,
	fonttitle = \bfseries\sffamily,
	description font = \mdseries,
	separator sign none,
	segmentation style={solid, myexercisefg},
}
{th}

\tcbuselibrary{theorems,skins,hooks}
\newtcbtheorem[number within=chapter]{exercise}{Exercise}
{%
	enhanced,
	breakable,
	colback = myexercisebg,
	frame hidden,
	boxrule = 0sp,
	borderline west = {2pt}{0pt}{myexercisefg},
	sharp corners,
	detach title,
	before upper = \tcbtitle\par\smallskip,
	coltitle = myexercisefg,
	fonttitle = \bfseries\sffamily,
	description font = \mdseries,
	separator sign none,
	segmentation style={solid, myexercisefg},
}
{th}


%================================
% PROPOSITION
%================================

\tcbuselibrary{theorems,skins,hooks}
\newtcbtheorem[number within=section]{Prop}{Proposition}
{%
	enhanced,
	breakable,
	colback = mypropbg,
	frame hidden,
	boxrule = 0sp,
	borderline west = {2pt}{0pt}{mypropfr},
	sharp corners,
	detach title,
	before upper = \tcbtitle\par\smallskip,
	coltitle = mypropfr,
	fonttitle = \bfseries\sffamily,
	description font = \mdseries,
	separator sign none,
	segmentation style={solid, mypropfr},
}
{th}

\tcbuselibrary{theorems,skins,hooks}
\newtcbtheorem[number within=chapter]{prop}{Proposition}
{%
	enhanced,
	breakable,
	colback = mypropbg,
	frame hidden,
	boxrule = 0sp,
	borderline west = {2pt}{0pt}{mypropfr},
	sharp corners,
	detach title,
	before upper = \tcbtitle\par\smallskip,
	coltitle = mypropfr,
	fonttitle = \bfseries\sffamily,
	description font = \mdseries,
	separator sign none,
	segmentation style={solid, mypropfr},
}
{th}


%================================
% CLAIM
%================================

\tcbuselibrary{theorems,skins,hooks}
\newtcbtheorem[number within=section]{claim}{Claim}
{%
	enhanced
	,breakable
	,colback = myg!10
	,frame hidden
	,boxrule = 0sp
	,borderline west = {2pt}{0pt}{myg}
	,sharp corners
	,detach title
	,before upper = \tcbtitle\par\smallskip
	,coltitle = myg!85!black
	,fonttitle = \bfseries\sffamily
	,description font = \mdseries
	,separator sign none
	,segmentation style={solid, myg!85!black}
}
{th}



%================================
% EXAMPLE BOX
%================================

\newtcbtheorem[number within=section]{Example}{Example}
{%
	colback = myexamplebg
	,breakable
	,colframe = myexamplefr
	,coltitle = myexampleti
	,boxrule = 1pt
	,sharp corners
	,detach title
	,before upper=\tcbtitle\par\smallskip
	,fonttitle = \bfseries
	,description font = \mdseries
	,separator sign none
	,description delimiters parenthesis
}
{ex}

\newtcbtheorem[number within=chapter]{example}{Example}
{%
	colback = myexamplebg
	,breakable
	,colframe = myexamplefr
	,coltitle = myexampleti
	,boxrule = 1pt
	,sharp corners
	,detach title
	,before upper=\tcbtitle\par\smallskip
	,fonttitle = \bfseries
	,description font = \mdseries
	,separator sign none
	,description delimiters parenthesis
}
{ex}

%================================
% DEFINITION BOX
%================================

\newtcbtheorem[number within=section]{Definition}{Definition}{enhanced,
	before skip=2mm,after skip=2mm, colback=red!5,colframe=red!80!black,boxrule=0.5mm,
	attach boxed title to top left={xshift=1cm,yshift*=1mm-\tcboxedtitleheight}, varwidth boxed title*=-3cm,
	boxed title style={frame code={
					\path[fill=tcbcolback]
					([yshift=-1mm,xshift=-1mm]frame.north west)
					arc[start angle=0,end angle=180,radius=1mm]
					([yshift=-1mm,xshift=1mm]frame.north east)
					arc[start angle=180,end angle=0,radius=1mm];
					\path[left color=tcbcolback!60!black,right color=tcbcolback!60!black,
						middle color=tcbcolback!80!black]
					([xshift=-2mm]frame.north west) -- ([xshift=2mm]frame.north east)
					[rounded corners=1mm]-- ([xshift=1mm,yshift=-1mm]frame.north east)
					-- (frame.south east) -- (frame.south west)
					-- ([xshift=-1mm,yshift=-1mm]frame.north west)
					[sharp corners]-- cycle;
				},interior engine=empty,
		},
	fonttitle=\bfseries,
	title={#2},#1}{def}
\newtcbtheorem[number within=chapter]{definition}{Definition}{enhanced,
	before skip=2mm,after skip=2mm, colback=red!5,colframe=red!80!black,boxrule=0.5mm,
	attach boxed title to top left={xshift=1cm,yshift*=1mm-\tcboxedtitleheight}, varwidth boxed title*=-3cm,
	boxed title style={frame code={
					\path[fill=tcbcolback]
					([yshift=-1mm,xshift=-1mm]frame.north west)
					arc[start angle=0,end angle=180,radius=1mm]
					([yshift=-1mm,xshift=1mm]frame.north east)
					arc[start angle=180,end angle=0,radius=1mm];
					\path[left color=tcbcolback!60!black,right color=tcbcolback!60!black,
						middle color=tcbcolback!80!black]
					([xshift=-2mm]frame.north west) -- ([xshift=2mm]frame.north east)
					[rounded corners=1mm]-- ([xshift=1mm,yshift=-1mm]frame.north east)
					-- (frame.south east) -- (frame.south west)
					-- ([xshift=-1mm,yshift=-1mm]frame.north west)
					[sharp corners]-- cycle;
				},interior engine=empty,
		},
	fonttitle=\bfseries,
	title={#2},#1}{def}


%================================
% EXERCISE BOX
%================================

\newcounter{questioncounter}
\counterwithin{questioncounter}{chapter}
% \counterwithin{questioncounter}{section}

\makeatletter
\newtcbtheorem[use counter=questioncounter]{question}{Question}{enhanced,
	breakable,
	colback=white,
	colframe=myb!80!black,
	attach boxed title to top left={yshift*=-\tcboxedtitleheight},
	fonttitle=\bfseries,
	title={#2},
	boxed title size=title,
	boxed title style={%
			sharp corners,
			rounded corners=northwest,
			colback=tcbcolframe,
			boxrule=0pt,
		},
	underlay boxed title={%
			\path[fill=tcbcolframe] (title.south west)--(title.south east)
			to[out=0, in=180] ([xshift=5mm]title.east)--
			(title.center-|frame.east)
			[rounded corners=\kvtcb@arc] |-
			(frame.north) -| cycle;
		},
	#1
}{def}
\makeatother

%================================
% SOLUTION BOX
%================================

\makeatletter
\newtcolorbox{solution}{enhanced,
	breakable,
	colback=white,
	colframe=myg!80!black,
	attach boxed title to top left={yshift*=-\tcboxedtitleheight},
	title=Solution,
	boxed title size=title,
	boxed title style={%
			sharp corners,
			rounded corners=northwest,
			colback=tcbcolframe,
			boxrule=0pt,
		},
	underlay boxed title={%
			\path[fill=tcbcolframe] (title.south west)--(title.south east)
			to[out=0, in=180] ([xshift=5mm]title.east)--
			(title.center-|frame.east)
			[rounded corners=\kvtcb@arc] |-
			(frame.north) -| cycle;
		},
}
\makeatother

%================================
% Question BOX
%================================

\makeatletter
\newtcbtheorem{qstion}{Question}{enhanced,
	breakable,
	colback=white,
	colframe=mygr,
	attach boxed title to top left={yshift*=-\tcboxedtitleheight},
	fonttitle=\bfseries,
	title={#2},
	boxed title size=title,
	boxed title style={%
			sharp corners,
			rounded corners=northwest,
			colback=tcbcolframe,
			boxrule=0pt,
		},
	underlay boxed title={%
			\path[fill=tcbcolframe] (title.south west)--(title.south east)
			to[out=0, in=180] ([xshift=5mm]title.east)--
			(title.center-|frame.east)
			[rounded corners=\kvtcb@arc] |-
			(frame.north) -| cycle;
		},
	#1
}{def}
\makeatother

\newtcbtheorem[number within=chapter]{wconc}{Wrong Concept}{
	breakable,
	enhanced,
	colback=white,
	colframe=myr,
	arc=0pt,
	outer arc=0pt,
	fonttitle=\bfseries\sffamily\large,
	colbacktitle=myr,
	attach boxed title to top left={},
	boxed title style={
			enhanced,
			skin=enhancedfirst jigsaw,
			arc=3pt,
			bottom=0pt,
			interior style={fill=myr}
		},
	#1
}{def}


%================================
% NOTE BOX
%================================

\usetikzlibrary{arrows,calc,shadows.blur}
\tcbuselibrary{skins}
\newtcolorbox{note}[1][]{%
	enhanced jigsaw,
	colback=gray!20!white,%
	colframe=gray!80!black,
	size=small,
	boxrule=1pt,
	title=\textbf{Note:-},
	halign title=flush center,
	coltitle=black,
	breakable,
	drop shadow=black!50!white,
	attach boxed title to top left={xshift=1cm,yshift=-\tcboxedtitleheight/2,yshifttext=-\tcboxedtitleheight/2},
	minipage boxed title=1.5cm,
	boxed title style={%
			colback=white,
			size=fbox,
			boxrule=1pt,
			boxsep=2pt,
			underlay={%
					\coordinate (dotA) at ($(interior.west) + (-0.5pt,0)$);
					\coordinate (dotB) at ($(interior.east) + (0.5pt,0)$);
					\begin{scope}
						\clip (interior.north west) rectangle ([xshift=3ex]interior.east);
						\filldraw [white, blur shadow={shadow opacity=60, shadow yshift=-.75ex}, rounded corners=2pt] (interior.north west) rectangle (interior.south east);
					\end{scope}
					\begin{scope}[gray!80!black]
						\fill (dotA) circle (2pt);
						\fill (dotB) circle (2pt);
					\end{scope}
				},
		},
	#1,
}

%%%%%%%%%%%%%%%%%%%%%%%%%%%%%%
% SELF MADE COMMANDS
%%%%%%%%%%%%%%%%%%%%%%%%%%%%%%

\newcommand{\thm}[2]{\begin{Theorem}{#1}{}#2\end{Theorem}}
\newcommand{\cor}[2]{\begin{Corollary}{#1}{}#2\end{Corollary}}
\newcommand{\mlemma}[2]{\begin{Lemma}{#1}{}#2\end{Lemma}}
\newcommand{\mer}[2]{\begin{Exercise}{#1}{}#2\end{Exercise}}
\newcommand{\mprop}[2]{\begin{Prop}{#1}{}#2\end{Prop}}
\newcommand{\clm}[3]{\begin{claim}{#1}{#2}#3\end{claim}}
\newcommand{\wc}[2]{\begin{wconc}{#1}{}\setlength{\parindent}{1cm}#2\end{wconc}}
\newcommand{\thmcon}[1]{\begin{Theoremcon}{#1}\end{Theoremcon}}
\newcommand{\ex}[2]{\begin{Example}{#1}{}#2\end{Example}}
\newcommand{\dfn}[2]{\begin{Definition}[colbacktitle=red!75!black]{#1}{}#2\end{Definition}}
\newcommand{\dfnc}[2]{\begin{definition}[colbacktitle=red!75!black]{#1}{}#2\end{definition}}
\newcommand{\qs}[2]{\begin{question}{#1}{}#2\end{question}}
\newcommand{\pf}[2]{\begin{myproof}[#1]#2\end{myproof}}
\newcommand{\nt}[1]{\begin{note}#1\end{note}}

\newcommand*\circled[1]{\tikz[baseline=(char.base)]{
		\node[shape=circle,draw,inner sep=1pt] (char) {#1};}}
\newcommand\getcurrentref[1]{%
	\ifnumequal{\value{#1}}{0}
	{??}
	{\the\value{#1}}%
}
\newcommand{\getCurrentSectionNumber}{\getcurrentref{section}}
\newenvironment{myproof}[1][\proofname]{%
	\proof[\bfseries #1: ]%
}{\endproof}

\newcommand{\mclm}[2]{\begin{myclaim}[#1]#2\end{myclaim}}
\newenvironment{myclaim}[1][\claimname]{\proof[\bfseries #1: ]}{}
\newenvironment{iclaim}[1][\claimname]{\bfseries #1\mdseries:}{}
\newcommand{\iclm}[2]{\begin{iclaim}[#1]#2\end{iclaim}}

\newcounter{mylabelcounter}

\makeatletter
\newcommand{\setword}[2]{%
	\phantomsection
	#1\def\@currentlabel{\unexpanded{#1}}\label{#2}%
}
\makeatother

% deliminators
\DeclarePairedDelimiter{\abs}{\lvert}{\rvert}
\DeclarePairedDelimiter{\norm}{\lVert}{\rVert}

\DeclarePairedDelimiter{\ceil}{\lceil}{\rceil}
\DeclarePairedDelimiter{\floor}{\lfloor}{\rfloor}
\DeclarePairedDelimiter{\round}{\lfloor}{\rceil}

\newsavebox\diffdbox
\newcommand{\slantedromand}{{\mathpalette\makesl{d}}}
\newcommand{\makesl}[2]{%
	\begingroup
	\sbox{\diffdbox}{$\mathsurround=0pt#1\mathrm{#2}$}%
	\pdfsave
	\pdfsetmatrix{1 0 0.2 1}%
	\rlap{\usebox{\diffdbox}}%
	\pdfrestore
	\hskip\wd\diffdbox
	\endgroup
}
\newcommand{\dd}[1][]{\ensuremath{\mathop{}\!\ifstrempty{#1}{%
			\slantedromand\@ifnextchar^{\hspace{0.2ex}}{\hspace{0.1ex}}}%
		{\slantedromand\hspace{0.2ex}^{#1}}}}
\ProvideDocumentCommand\dv{o m g}{%
	\ensuremath{%
		\IfValueTF{#3}{%
			\IfNoValueTF{#1}{%
				\frac{\dd #2}{\dd #3}%
			}{%
				\frac{\dd^{#1} #2}{\dd #3^{#1}}%
			}%
		}{%
			\IfNoValueTF{#1}{%
				\frac{\dd}{\dd #2}%
			}{%
				\frac{\dd^{#1}}{\dd #2^{#1}}%
			}%
		}%
	}%
}
\providecommand*{\pdv}[3][]{\frac{\partial^{#1}#2}{\partial#3^{#1}}}
%  - others
\DeclareMathOperator{\Lap}{\mathcal{L}}
\DeclareMathOperator{\Var}{Var} % varience
\DeclareMathOperator{\Cov}{Cov} % covarience
\DeclareMathOperator{\E}{E} % expected

% Since the amsthm package isn't loaded

% I prefer the slanted \leq
\let\oldleq\leq % save them in case they're every wanted
\let\oldgeq\geq
\renewcommand{\leq}{\leqslant}
\renewcommand{\geq}{\geqslant}

%%%%%%%%%%%%%%%%%%%%%%%%%%%%%%%%%%%%%%%%%%%
% TABLE OF CONTENTS
%%%%%%%%%%%%%%%%%%%%%%%%%%%%%%%%%%%%%%%%%%%

\contentsmargin{0cm}
\titlecontents{chapter}[3.7pc]
{\addvspace{30pt}%
	\begin{tikzpicture}[remember picture, overlay]%
		\draw[fill=doc!60,draw=doc!60] (-7,-.1) rectangle (-0.9,.5);%
		\pgftext[left,x=-3.7cm,y=0.2cm]{\color{white}\Large\sc\bfseries Chapter\ \thecontentslabel};%
	\end{tikzpicture}\color{doc!60}\large\sc\bfseries}%
{}
{}
{\;\titlerule\;\large\sc\bfseries Page \thecontentspage
	\begin{tikzpicture}[remember picture, overlay]
		\draw[fill=doc!60,draw=doc!60] (2pt,0) rectangle (4,0.1pt);
	\end{tikzpicture}}%
\titlecontents{section}[3.7pc]
{\addvspace{2pt}}
{\contentslabel[\thecontentslabel]{2pc}}
{}
{\hfill\small \thecontentspage}
[]
\titlecontents*{subsection}[3.7pc]
{\addvspace{-1pt}\small}
{}
{}
{\ --- \small\thecontentspage}
[ \textbullet\ ][]

\makeatletter
\renewcommand{\tableofcontents}{%
	\chapter*{%
	  \vspace*{-20\p@}%
	  \begin{tikzpicture}[remember picture, overlay]%
		  \pgftext[right,x=15cm,y=0.2cm]{\color{doc!60}\Huge\sc\bfseries \contentsname};%
		  \draw[fill=doc!60,draw=doc!60] (13,-.75) rectangle (20,1);%
		  \clip (13,-.75) rectangle (20,1);
		  \pgftext[right,x=15cm,y=0.2cm]{\color{white}\Huge\sc\bfseries \contentsname};%
	  \end{tikzpicture}}%
	\@starttoc{toc}}
\makeatother

\newcommand{\eps}{\epsilon}
\newcommand{\veps}{\varepsilon}
\newcommand{\Qed}{\begin{flushright}\qed\end{flushright}}

\newcommand{\parinn}{\setlength{\parindent}{1cm}}
\newcommand{\parinf}{\setlength{\parindent}{0cm}}

% \newcommand{\norm}{\|\cdot\|}
\newcommand{\inorm}{\norm_{\infty}}
\newcommand{\opensets}{\{V_{\alpha}\}_{\alpha\in I}}
\newcommand{\oset}{V_{\alpha}}
\newcommand{\opset}[1]{V_{\alpha_{#1}}}
\newcommand{\lub}{\text{lub}}
\newcommand{\del}[2]{\frac{\partial #1}{\partial #2}}
\newcommand{\Del}[3]{\frac{\partial^{#1} #2}{\partial^{#1} #3}}
\newcommand{\deld}[2]{\dfrac{\partial #1}{\partial #2}}
\newcommand{\Deld}[3]{\dfrac{\partial^{#1} #2}{\partial^{#1} #3}}
\newcommand{\der}[2]{\frac{\mathrm{d} #1}{\mathrm{d} #2}}
% \newcommand{\ddd}[3]{\frac{\mathrm{d}^{#3} #1}{\mathrm{d}^{#3} #2}}
\newcommand{\lm}{\lambda}
\newcommand{\uin}{\mathbin{\rotatebox[origin=c]{90}{$\in$}}}
\newcommand{\usubset}{\mathbin{\rotatebox[origin=c]{90}{$\subset$}}}
\newcommand{\lt}{\left}
\newcommand{\rt}{\right}
\newcommand{\bs}[1]{\boldsymbol{#1}}
\newcommand{\exs}{\exists}
\newcommand{\st}{\strut}
\newcommand{\dps}[1]{\displaystyle{#1}}
\newcommand{\id}{\text{id}}


\newcommand{\sol}{\setlength{\parindent}{0cm}\textbf{\textit{Solution:}}\setlength{\parindent}{1cm} }
\newcommand{\solve}[1]{\setlength{\parindent}{0cm}\textbf{\textit{Solution: }}\setlength{\parindent}{1cm}#1 \Qed}

% number sets
\newcommand{\RR}[1][]{\ensuremath{\ifstrempty{#1}{\mathbb{R}}{\mathbb{R}^{#1}}}}
\newcommand{\NN}[1][]{\ensuremath{\ifstrempty{#1}{\mathbb{N}}{\mathbb{N}^{#1}}}}
\newcommand{\ZZ}[1][]{\ensuremath{\ifstrempty{#1}{\mathbb{Z}}{\mathbb{Z}^{#1}}}}
\newcommand{\QQ}[1][]{\ensuremath{\ifstrempty{#1}{\mathbb{Q}}{\mathbb{Q}^{#1}}}}
\newcommand{\CC}[1][]{\ensuremath{\ifstrempty{#1}{\mathbb{C}}{\mathbb{C}^{#1}}}}
\newcommand{\PP}[1][]{\ensuremath{\ifstrempty{#1}{\mathbb{P}}{\mathbb{P}^{#1}}}}
\newcommand{\HH}[1][]{\ensuremath{\ifstrempty{#1}{\mathbb{H}}{\mathbb{H}^{#1}}}}
\newcommand{\FF}[1][]{\ensuremath{\ifstrempty{#1}{\mathbb{F}}{\mathbb{F}^{#1}}}}
% expected value
\newcommand{\EE}{\ensuremath{\mathbb{E}}}

%---------------------------------------
% BlackBoard Math Fonts :-
%---------------------------------------

%Captital Letters
\newcommand{\bbA}{\mathbb{A}}	\newcommand{\bbB}{\mathbb{B}}
\newcommand{\bbC}{\mathbb{C}}	\newcommand{\bbD}{\mathbb{D}}
\newcommand{\bbE}{\mathbb{E}}	\newcommand{\bbF}{\mathbb{F}}
\newcommand{\bbG}{\mathbb{G}}	\newcommand{\bbH}{\mathbb{H}}
\newcommand{\bbI}{\mathbb{I}}	\newcommand{\bbJ}{\mathbb{J}}
\newcommand{\bbK}{\mathbb{K}}	\newcommand{\bbL}{\mathbb{L}}
\newcommand{\bbM}{\mathbb{M}}	\newcommand{\bbN}{\mathbb{N}}
\newcommand{\bbO}{\mathbb{O}}	\newcommand{\bbP}{\mathbb{P}}
\newcommand{\bbQ}{\mathbb{Q}}	\newcommand{\bbR}{\mathbb{R}}
\newcommand{\bbS}{\mathbb{S}}	\newcommand{\bbT}{\mathbb{T}}
\newcommand{\bbU}{\mathbb{U}}	\newcommand{\bbV}{\mathbb{V}}
\newcommand{\bbW}{\mathbb{W}}	\newcommand{\bbX}{\mathbb{X}}
\newcommand{\bbY}{\mathbb{Y}}	\newcommand{\bbZ}{\mathbb{Z}}

%---------------------------------------
% MathCal Fonts :-
%---------------------------------------

%Captital Letters
\newcommand{\mcA}{\mathcal{A}}	\newcommand{\mcB}{\mathcal{B}}
\newcommand{\mcC}{\mathcal{C}}	\newcommand{\mcD}{\mathcal{D}}
\newcommand{\mcE}{\mathcal{E}}	\newcommand{\mcF}{\mathcal{F}}
\newcommand{\mcG}{\mathcal{G}}	\newcommand{\mcH}{\mathcal{H}}
\newcommand{\mcI}{\mathcal{I}}	\newcommand{\mcJ}{\mathcal{J}}
\newcommand{\mcK}{\mathcal{K}}	\newcommand{\mcL}{\mathcal{L}}
\newcommand{\mcM}{\mathcal{M}}	\newcommand{\mcN}{\mathcal{N}}
\newcommand{\mcO}{\mathcal{O}}	\newcommand{\mcP}{\mathcal{P}}
\newcommand{\mcQ}{\mathcal{Q}}	\newcommand{\mcR}{\mathcal{R}}
\newcommand{\mcS}{\mathcal{S}}	\newcommand{\mcT}{\mathcal{T}}
\newcommand{\mcU}{\mathcal{U}}	\newcommand{\mcV}{\mathcal{V}}
\newcommand{\mcW}{\mathcal{W}}	\newcommand{\mcX}{\mathcal{X}}
\newcommand{\mcY}{\mathcal{Y}}	\newcommand{\mcZ}{\mathcal{Z}}



%---------------------------------------
% Bold Math Fonts :-
%---------------------------------------

%Captital Letters
\newcommand{\bmA}{\boldsymbol{A}}	\newcommand{\bmB}{\boldsymbol{B}}
\newcommand{\bmC}{\boldsymbol{C}}	\newcommand{\bmD}{\boldsymbol{D}}
\newcommand{\bmE}{\boldsymbol{E}}	\newcommand{\bmF}{\boldsymbol{F}}
\newcommand{\bmG}{\boldsymbol{G}}	\newcommand{\bmH}{\boldsymbol{H}}
\newcommand{\bmI}{\boldsymbol{I}}	\newcommand{\bmJ}{\boldsymbol{J}}
\newcommand{\bmK}{\boldsymbol{K}}	\newcommand{\bmL}{\boldsymbol{L}}
\newcommand{\bmM}{\boldsymbol{M}}	\newcommand{\bmN}{\boldsymbol{N}}
\newcommand{\bmO}{\boldsymbol{O}}	\newcommand{\bmP}{\boldsymbol{P}}
\newcommand{\bmQ}{\boldsymbol{Q}}	\newcommand{\bmR}{\boldsymbol{R}}
\newcommand{\bmS}{\boldsymbol{S}}	\newcommand{\bmT}{\boldsymbol{T}}
\newcommand{\bmU}{\boldsymbol{U}}	\newcommand{\bmV}{\boldsymbol{V}}
\newcommand{\bmW}{\boldsymbol{W}}	\newcommand{\bmX}{\boldsymbol{X}}
\newcommand{\bmY}{\boldsymbol{Y}}	\newcommand{\bmZ}{\boldsymbol{Z}}
%Small Letters
\newcommand{\bma}{\boldsymbol{a}}	\newcommand{\bmb}{\boldsymbol{b}}
\newcommand{\bmc}{\boldsymbol{c}}	\newcommand{\bmd}{\boldsymbol{d}}
\newcommand{\bme}{\boldsymbol{e}}	\newcommand{\bmf}{\boldsymbol{f}}
\newcommand{\bmg}{\boldsymbol{g}}	\newcommand{\bmh}{\boldsymbol{h}}
\newcommand{\bmi}{\boldsymbol{i}}	\newcommand{\bmj}{\boldsymbol{j}}
\newcommand{\bmk}{\boldsymbol{k}}	\newcommand{\bml}{\boldsymbol{l}}
\newcommand{\bmm}{\boldsymbol{m}}	\newcommand{\bmn}{\boldsymbol{n}}
\newcommand{\bmo}{\boldsymbol{o}}	\newcommand{\bmp}{\boldsymbol{p}}
\newcommand{\bmq}{\boldsymbol{q}}	\newcommand{\bmr}{\boldsymbol{r}}
\newcommand{\bms}{\boldsymbol{s}}	\newcommand{\bmt}{\boldsymbol{t}}
\newcommand{\bmu}{\boldsymbol{u}}	\newcommand{\bmv}{\boldsymbol{v}}
\newcommand{\bmw}{\boldsymbol{w}}	\newcommand{\bmx}{\boldsymbol{x}}
\newcommand{\bmy}{\boldsymbol{y}}	\newcommand{\bmz}{\boldsymbol{z}}

%---------------------------------------
% Scr Math Fonts :-
%---------------------------------------

\newcommand{\sA}{{\mathscr{A}}}   \newcommand{\sB}{{\mathscr{B}}}
\newcommand{\sC}{{\mathscr{C}}}   \newcommand{\sD}{{\mathscr{D}}}
\newcommand{\sE}{{\mathscr{E}}}   \newcommand{\sF}{{\mathscr{F}}}
\newcommand{\sG}{{\mathscr{G}}}   \newcommand{\sH}{{\mathscr{H}}}
\newcommand{\sI}{{\mathscr{I}}}   \newcommand{\sJ}{{\mathscr{J}}}
\newcommand{\sK}{{\mathscr{K}}}   \newcommand{\sL}{{\mathscr{L}}}
\newcommand{\sM}{{\mathscr{M}}}   \newcommand{\sN}{{\mathscr{N}}}
\newcommand{\sO}{{\mathscr{O}}}   \newcommand{\sP}{{\mathscr{P}}}
\newcommand{\sQ}{{\mathscr{Q}}}   \newcommand{\sR}{{\mathscr{R}}}
\newcommand{\sS}{{\mathscr{S}}}   \newcommand{\sT}{{\mathscr{T}}}
\newcommand{\sU}{{\mathscr{U}}}   \newcommand{\sV}{{\mathscr{V}}}
\newcommand{\sW}{{\mathscr{W}}}   \newcommand{\sX}{{\mathscr{X}}}
\newcommand{\sY}{{\mathscr{Y}}}   \newcommand{\sZ}{{\mathscr{Z}}}


%---------------------------------------
% Math Fraktur Font
%---------------------------------------

%Captital Letters
\newcommand{\mfA}{\mathfrak{A}}	\newcommand{\mfB}{\mathfrak{B}}
\newcommand{\mfC}{\mathfrak{C}}	\newcommand{\mfD}{\mathfrak{D}}
\newcommand{\mfE}{\mathfrak{E}}	\newcommand{\mfF}{\mathfrak{F}}
\newcommand{\mfG}{\mathfrak{G}}	\newcommand{\mfH}{\mathfrak{H}}
\newcommand{\mfI}{\mathfrak{I}}	\newcommand{\mfJ}{\mathfrak{J}}
\newcommand{\mfK}{\mathfrak{K}}	\newcommand{\mfL}{\mathfrak{L}}
\newcommand{\mfM}{\mathfrak{M}}	\newcommand{\mfN}{\mathfrak{N}}
\newcommand{\mfO}{\mathfrak{O}}	\newcommand{\mfP}{\mathfrak{P}}
\newcommand{\mfQ}{\mathfrak{Q}}	\newcommand{\mfR}{\mathfrak{R}}
\newcommand{\mfS}{\mathfrak{S}}	\newcommand{\mfT}{\mathfrak{T}}
\newcommand{\mfU}{\mathfrak{U}}	\newcommand{\mfV}{\mathfrak{V}}
\newcommand{\mfW}{\mathfrak{W}}	\newcommand{\mfX}{\mathfrak{X}}
\newcommand{\mfY}{\mathfrak{Y}}	\newcommand{\mfZ}{\mathfrak{Z}}
%Small Letters
\newcommand{\mfa}{\mathfrak{a}}	\newcommand{\mfb}{\mathfrak{b}}
\newcommand{\mfc}{\mathfrak{c}}	\newcommand{\mfd}{\mathfrak{d}}
\newcommand{\mfe}{\mathfrak{e}}	\newcommand{\mff}{\mathfrak{f}}
\newcommand{\mfg}{\mathfrak{g}}	\newcommand{\mfh}{\mathfrak{h}}
\newcommand{\mfi}{\mathfrak{i}}	\newcommand{\mfj}{\mathfrak{j}}
\newcommand{\mfk}{\mathfrak{k}}	\newcommand{\mfl}{\mathfrak{l}}
\newcommand{\mfm}{\mathfrak{m}}	\newcommand{\mfn}{\mathfrak{n}}
\newcommand{\mfo}{\mathfrak{o}}	\newcommand{\mfp}{\mathfrak{p}}
\newcommand{\mfq}{\mathfrak{q}}	\newcommand{\mfr}{\mathfrak{r}}
\newcommand{\mfs}{\mathfrak{s}}	\newcommand{\mft}{\mathfrak{t}}
\newcommand{\mfu}{\mathfrak{u}}	\newcommand{\mfv}{\mathfrak{v}}
\newcommand{\mfw}{\mathfrak{w}}	\newcommand{\mfx}{\mathfrak{x}}
\newcommand{\mfy}{\mathfrak{y}}	\newcommand{\mfz}{\mathfrak{z}}

\title{\Huge{MATH 231 : Numerical ODEs}}
\author{\huge{Pratham Lalwani}}
\date{\today}

\begin{document}
\maketitle
\qs{Eigenvalues of special tridiagonal matrices}{

	This question is about finding eigenvalues of tridiagonal linear systems arising from applications, specifically finding the eigenvalues of an $n \times n$ matrix of the form,

	$$
		A=\left(\begin{array}{ccccc}
			a & b      &        &        &   \\
			c & a      & b      &        &   \\
			  & \ddots & \ddots & \ddots &   \\
			  &        & c      & a      & b \\
			  &        &        & c      & a
		\end{array}\right)
	$$

	where $a, b, c$ are real numbers with $b c>0$ (i.e. $b$ and $c$ have the same signs).
	\begin{enumerate}[label=(\alph*)]
		\item Show that the eigenvalue problem of $A$ is equivalent to the equations

		      $$
			      \begin{aligned}
				      c v_{j-1}+(a-\lambda) v_j+b v_{j+1} & =0, \quad j=1, \ldots, n \\
				      v_0                                 & =0=v_{n+1}
			      \end{aligned}
		      $$

		      where $\boldsymbol{v}=\left(v_1, \ldots, v_n\right)^T$ is an eigenvector of $A$ associated with the eigenvalue $\lambda$.
		\item The recurrence relation (1) is a second order linear difference equation and can be solved similar to second order linear differential equations. By guessing $v_j=r^j$ for some constant $r$, show that $r$ satisfies

		      $$
			      r_{ \pm}=\frac{\lambda-a \pm \sqrt{(\lambda-a)^2-4 b c}}{2 b}, \quad \text { with } \quad r_{+} r_{-}=\frac{c}{b}
		      $$

		\item Show by contradiction that $r_{ \pm}$must be distinct.

		      Hint: if $r_{ \pm}=r$ are repeated, then $v_j=A r^j+B j r^j$ for some constants $A, B$.
		\item Since $r_{ \pm}$are distinct, the general solution for (1) is $v_j=A r_{+}^j+B r_{-}^j$ for constants $A, B$. Use this to conclude from (2) and (3) that,
		      $$
			      \left(\frac{b r_{+}^2}{c}\right)^{n+1}=1
		      $$

		\item From part (c), (3) and (4), show that $r_{ \pm}$must be complex valued and conclude that (4) has the solutions for $k=1, \ldots, n$,

		      $$
			      r_{ \pm, k}=\sqrt{\frac{c}{b}} \exp \left(\frac{ \pm i k \pi}{n+1}\right), \quad \text { where } i=\sqrt{-1}
		      $$

		\item Using part (e), conclude that the eigenvalues of $A$ is given by

		      $$
			      \lambda_k=a+2 \operatorname{sgn}\left(b  \right) \sqrt{bc} \cos \left(\frac{\pi k}{n+1}\right), \quad k=1, \ldots, n
		      $$
		\item
		      Find the eigenvalues of the $n \times n$ finite difference matrix $A_h=\frac{1}{h^2}\left(\begin{array}{ccccc}2 & -1 & && \\ -1 & 2 & \ddots&&  \\ & \ddots & \ddots \\ &&-1&2&-1 \\   &&\\ &&&-1&2 \end{array}\right)$, where $h=\frac{1}{n+1}$.

		      Conclude that $A_h$ is symmetric positive definite and find its condition number $\kappa\left(A_h\right)$ with respect to $\|\cdot\|_2$. Show that $\kappa\left(A_h\right)=\mathcal{O}\left(h^{-2}\right)$ as number of grid points $n$ increases. What does this mean for solving $A_h \boldsymbol{x}=\boldsymbol{b}$ when $n$ is large?

	\end{enumerate}
}
\begin{solution}
	\begin{enumerate}[label=(\alph*)]
		\item Let ($\lambda, \vec{v}$) be an eigenpair of $A$
		      \begin{align*}
			      A\vec{v}                                 & =   \lambda \vec{v} \\
			      \left( A - \lambda I \right)\vec{v}      & = \vec{0} \\
			      \left(\begin{array}{ccccc}
					            \left( a-\lambda \right) v_1 + bv_2   \\
					            cv_1 + (a-\lambda)v_2 + bv_3          \\
					            \vdots                                \\
					            cv_{n-2} + (a-\lambda) v_{n-1}+ b_{n} \\
					            c_{n-1} + (a-\lambda) v_{n}
				            \end{array}   \right) & = \vec{0} . \\
		      \end{align*}
		      We can write the above relation as the following,
		      \begin{equation}
			      cv_{j-1} + \left( a-\lambda \right) v_{j} + bv_{j+1} = 0  \label{req_rel}
			      . \end{equation}
		      Where  $0\le j \le n+1$ and $v_{0} = 0 = v_{n+1}$
		      \qed
		\item Using the hint we guess the following form of the solution $v_{j} = r^j$. Substituting in \ref{req_rel},
		      \begin{align*}
			      cr^{j-1} + \left( a-\lambda \right) r^{j} + br^{j+1} & =   0 \\
			      c + \left( a-\lambda \right) r + br^{2}              & =   0 \\
			      .\end{align*}
		      Using the quadratic formula, we get
		      \[
			      r_{\pm} = \frac{\lambda -a \pm \sqrt{(a-\lambda)^2 - 4bc} }{ 2b}.
		      \]
		      As $r_{\pm}$ are the roots to a quadratic, hence
		      \begin{equation}
			      r_{+}r_{-} = \dfrac{c}{b} \label{prod_rs}
		      \end{equation}
		\item If \ref{req_rel} has a repeated root, say $r_{\pm} = r $ , then solution to the recursion would look like,
		      \[
			      v_{j} = Ar^j + Bjr^j
			      .\]
		      Checking the boundary conditions,
		      $v_{0} = 0= v_{n+1}$
		      \begin{equation}
			      v_0 = A r^0 + B(0)r^{0} = A = 0 \label{v0}
			      .\end{equation}
		      \begin{equation}
			      v_{n+1} = (0) r^{n+1} + B(n+1)r^{n+1} = B(n+1)r^{n+1} = 0      \implies B =0 \label{vn+1}
			      .\end{equation}
		      Combining \ref{v0} \& \ref{vn+1} gives,
		      \[
			      v_j  =0
			      .\]
		      Which is the trivial eigenvector. Hence, we cannot have a repeated root if we want a non-zero eigenvector.
		\item From (c) we have that roots are distinct. Therefore, we look for solutions of the form $v_{j} = Ar_{+}^j + Br_{-}^j$ for some constants A and B defined by the "boundary conditions" of the recursion.
		      We have,
		      \begin{align}
			      v_{0}   & = A+B = 0 \implies A = -B    \nonumber \\
			      v_{n+1} & =  Ar_{+}^{n+1}  + Br_{-}^{n+1} =0 \implies r_{+}^{n+1} = r_{-}^{n+1} \label{rs_rel}
		      \end{align}
		      From, \ref{prod_rs} and \ref{rs_rel}, it follows that
		      \begin{align}
			      \left( r_{+}^{2} \right)^{(n+1)}          & =  \left( \frac{c}{b} \right)^{n+1} \nonumber \\
			      \left( \frac{br_{+}^2}{c} \right)^{(n+1)} & = 1 \label{rel_bcr}
		      \end{align}
		\item  We can observe in \ref{rel_bcr} that $\frac{br_{+}^{2}}{c}$ are the roots of unity, therefore,
		      \[
			      \frac{br_{+}^{2}}{c} = \exp\left( \frac{ik\pi}{n+1} \right) \implies r_{+} =  \sqrt{\frac{c}{b}} \exp\left( \frac{ik\pi}{n+1} \right) \quad k =0,\dots, n+1
			      .\]
		      Similarly,
		      \[
			      \frac{br_{-}^{2}}{c} = \exp\left( \frac{im\pi}{n+1} \right) \implies r_{-} =  \sqrt{\frac{c}{b}} \exp\left( \frac{im\pi}{n+1} \right) \quad m =0,\dots, n+1
			      .\]
		      Using \ref{prod_rs},
		      \[
			      r_{+}r_{-} =\frac{c}{b} \exp\left( \frac{i\left( k+m \right) \pi}{n+1} \right) =
			      .\]
	\end{enumerate}
\end{solution}


\qs{ Classical iterative methods for strictly diagonally dominant matrices}{
	\begin{enumerate}[label=(\alph*)]
		\item Show that the diagonal part of any strictly diagonally dominant (S.D.D.) matrix is invertible.
		\item Recall the Gershgorin's theorem below, which can give useful information about the eigenvalues of a matrix. The eigenvalues of a complex valued matrix $A$ lies in the union of $n$ discs $\bigcup_{i=1}^n D_i$ on the complex plane, where
		      $$
			      D_i=\left\{z \in \mathbb{C}:\left|z-a_{i i}\right| \leq \sum_{j \neq i}\left|a_{i j}\right|\right\}
		      $$
		      Using Gershgorin's theorem, conclude S.D.D. matrices are invertible. Hint: Show that $0 \notin D_i$ for all $i=1, \ldots, n$.

		      The next two parts are about showing convergence of Jacobi and Gauss-Seidel iterations for S.D.D. matrices.
		\item Recall the matrix $-M^{-1} N$ associated with the Jacobi iteration takes the form $-D^{-1}(L+U)$, where $A=L+D+U$.
		      \begin{enumerate}[label=(\roman*)]
			      \item Let $A$ be S.D.D. and $\lambda$ be any eigenvalue of $-D^{-1}(L+U)$. Show that $\operatorname{det}(L+U+\lambda D)=0$ using part (a).
			      \item Now suppose $|\lambda| \geq 1$. Deduce from $A$ being S.D.D. that $L+U+\lambda D$ must also be S.D.D.
			      \item Deduce a contradiction by applying the result from part (b) to $L+U+\lambda D$, and conclude that $|\lambda|<1$.
			      \item Combine parts (i)-(iii) to conclude that Jacobi iteration converges for S.D.D. matrices.
		      \end{enumerate}
		\item Follow a similar argument as part (c) to show that Gauss-Seidel iterations converges for S.D.D. matrices.

	\end{enumerate}
}
\begin{solution}

	\begin{enumerate}[label=(\alph*)]
		\item Let $A $ be a S.D.D matrix and, 		      \[
			      \implies a_{ii} > \sum_{\substack{j=1\\j\neq i}}^{n} a_{ij} \ge  0 \implies a_{ii} >0 \quad \forall 1\le i \le n \\
			      .\]
		      Let $D$ be the matrix containing the diagonial entries of $A$, hence
		      \[
			      D =    \begin{bmatrix} a_{11} &        &        \\
                       & \ddots &        \\
                       &        & a_{nn}
			      \end{bmatrix}
			      .\]
		      As, all $a_{ii}>0$, therefore we $D^{-1}$ exists.
		\item Let $\lambda_i$ be the eigenvalues associated with disc $D_i$.\\
		      Suppose $0\in D_{i}$ for some $1\le i \le n $, therefore, we have,
		      \[
			      a_{ii} \leq \sum_{\substack{ j=0 \\ j\neq i }}^{n} a_{ij}
			      .\]
		      Which is false as $A$ is a S.D.D matrix, hence , $0 \not\in D_{i}$.\\
		      Therefore we have, $|\lambda_i| >0 \quad \forall i, 1\le i\le n \implies A^{-1}$ exists.
		\item \begin{enumerate}[label=(\roman*)]
			      \item Given that $\lambda$ is an eigenvalue of $-D^{-1}\left( L+U \right)$. Therefore we have $\vec v $ such that $\vec v \neq 0$,
			            \begin{align*}
				            -D^{-1}\left( L+U \right)\vec{v}  & =   \lambda \vec{{v}} \\
				            \left(L+U \right) \vec v          & =   -\lambda D\vec{{v}} \\
				            \left(L+U +\lambda D\right)\vec v & =  \vec{0}
				            .\end{align*}
			            As there is a non-zero null vector associated with $L+U+\lambda D $, therefore $\det(L+U+\lambda D) = 0$.
			      \item Given that $A$ is S.D.D.  Suppose $|\lambda| \ge 1$. Consider,
			            \begin{align*}
				            |(L+U+\lambda D )_{ii}|  = |\lambda a_{ii}| & =  |\lambda| | a_{ii}| \\
				                                                        & >  |\lambda| | \sum_{\substack{j=1 \\j\neq i}}^{n} a_{ij} | \\
				                                                        & \ge | \sum_{\substack{j=1          \\j\neq i}}^{n} a_{ij} | \\
				                                                        & = | \sum_{\substack{j=1            \\j\neq i}}^{n} (L+U+\lambda D )_{ij} | \\
			            \end{align*}
			            Hence, $\left( L+U+\lambda D  \right) $ is S.D.D. .
			      \item If $|\lambda| \ge  1 $ and $A$ is S.D.D, gives that $(L+U+\lambda D)$  is S.D.D .

			            Therefore, $(L+U+\lambda D )$ is invertible. Which is a contradiction as $det(L+U+\lambda D)= 0$. Therefore, $|\lambda| < 1$.
			      \item Let $M = D$ and $N=L+U$. From parts (i)-(iii) we get,
			            \[
				            \lambda_{i} \leq \lambda_{max} < 1 \implies \rho(-M^{-1}N) < 1
				            .\]
			            By theorem of convergence of iterative solvers we get, iterations based on $-M^{-1}N$ converges to $0$.

		      \end{enumerate}
		\item

		      \begin{enumerate} [label=(\roman*)]
			      \item Let $\lambda$ be an eigenvalue of $-(L+D)^{-1}\left( U \right)$. Therefore we have $\vec v $ such that $\vec v \neq 0$,
			            \begin{align*}
				            -(L+D)^{-1}\left( L+U \right)\vec{v} & =   \lambda \vec{{v}} \\
				            \left(U \right) \vec v               & =   -\lambda (L+D)\vec{{v}} \\
				            \left(U +\lambda (L+D)\right)\vec v  & =  \vec{0}
				            .\end{align*}
			            As there is a non-zero null vector associated with $U+\lambda \left( L+D \right)  $, therefore $\det(U+\lambda\left( L+D \right) ) = 0$.
			      \item Given that $A$ is S.D.D.  Suppose $|\lambda| \ge 1$. Consider,
			            \begin{align*}
				            |(U+\lambda(L+D) )_{ii}|  = |\lambda a_{ii}| & =  |\lambda| | a_{ii}| \\
				                                                         & >  |\lambda| | \sum_{\substack{j=1 \\j\neq i}}^{n} a_{ij} | \\
				                                                         & \ge |\lambda \sum_{j=1}^{i-1} a_{ij}  +  \sum_{j=i+1}^{n} a_{ij}  | \\
				                                                         & = | \sum_{\substack{j=1 \\j\neq i}}^{n} (U+\lambda (L+D) )_{ij} | \\
			            \end{align*}
			            Hence, $\left( U+\lambda(L+D)  \right) $ is S.D.D. .
			      \item If $|\lambda| \ge  1 $ and $A$ is S.D.D, gives that $(U+\lambda (L+D))$  is S.D.D .

			            Therefore, $(L+U+\lambda (L+D) )$ is invertible. Which is a contradiction as $det(U+\lambda (L+D))= 0$. Therefore, $|\lambda| < 1$.

			      \item Let $M =L+ D$ and $N=U$. From parts (i)-(iii) we get,
			            \[
				            \lambda_{i} \leq \lambda_{max} < 1 \implies \rho(-M^{-1}N) < 1
				            .\]
			            By theorem of convergence of iterative solvers we get, iterations based on $-M^{-1}N$ converges to $0$.

		      \end{enumerate}


	\end{enumerate}
\end{solution}

\qs{Classical iterative methods for symmetric positive definite matrices
}{

	This question is about coding and comparing classical iterative methods for the S.P.D. matrix $A_h$ from Q1(g).
	\begin{enumerate}[label=(\alph*)]
		\item Write a pseudocode for the classical iterative methods: Richardson, optimal Richardson, Jacobi, Gauss-Seidel, S.O.R., and optimal S.O.R.
		\item Implement a program to solve $A_h \boldsymbol{x}=\boldsymbol{b}$ with $\boldsymbol{b}=(1, \ldots, 1)^T \in \mathbb{R}^{20}$ and $\boldsymbol{x}_0=(1,0, \ldots, 0)^T \in \mathbb{R}^{20}$ using Richardson (with $\omega=\lambda_{\max }^{-1}$ ), optimal Richardson, Jacobi, Gauss-Seidel, S.O.R. (with $\theta=1.2$ ) and optimal S.O.R. Generate a plot comparing the log of their residual in $\ell_2$ norm versus iterations up to 5000 . Rank the performance of each method by comparing the iterations needed to reach the residual tolerance of $10^{-14}$. Use sparse representation when appropriate.\\ \textit{ Hint: Use Q1(g) to find parameters for Richardson and vary $\theta$ to find an approximate optimal parameter for S.O.R. }
		\item  Comment on the decreases in performance when $n=1000$. Explain briefly how this relates to $\kappa\left(A_h\right)=\mathcal{O}\left(h^{-2}\right)$.
	\end{enumerate}
}
\begin{solution}

	\begin{algorithm}[H]
		\Fn{\FuncSty{RichardsonIteration(}{A,b,$\bm x_0$,$\omega$,tol,maxIter }\FuncSty{)}}{
			\KwIn{\\A: The matrix to find the solution to \\  $\bm b$: The resultant vector in $A\bm x = \bm b$  \\ $\bm x_0$: The initial guess \\ $\omega$: Richardson parameter (fixed)  \\ maxIter: The maximum of iterations}
			\KwOut{x: The solution to $A\bm x = \bm b$}
			\SetAlgoLined
			\SetNoFillComment
			\vspace{3mm}
			$M \leftarrow  \omega^{-1}I$ \\
			$N \leftarrow A- M$\\
			$x  \leftarrow x_0$\\
			$\bm r  \leftarrow \bm b-A\bm x  $\\
			\While{$\|r\|_2 < tol$ \textbf{and} $i <$ maxIter } {
				$\bm x  \leftarrow \bm x + \omega \bm r  $\\
				$\bm r  \leftarrow \bm b - A\bm x$
			}
			\Return $\bm x$;
			\caption{Richardson Iteration}
		}
	\end{algorithm}

	\begin{algorithm}[H]
		\Fn(){\FuncSty{OptimalRichardsonIteration(}{A,$\bm b$,$\bm x_0$,tol,maxIter }\FuncSty{)}}{
			\KwIn{\\A: The matrix to find the solution to \\  $\bm b$: The resultant vector in $A\bm x = \bm b$  \\ $\bm x_0$: The initial guess \\  maxIter: The maximum of iterations}
			\KwOut{x: The solution to $A\bm x = \bm b$}
			\SetAlgoLined
			\SetNoFillComment
			\vspace{3mm}
			$\omega \leftarrow \dfrac{2}{\lambda_{max}\left(A\right) + \lambda_{min}\left(A\right) }$ \\
			\vspace{1mm}
			$M \leftarrow \omega^{-1}I $ \\
			$N \leftarrow A- M$\\
			$x  \leftarrow x_0$\\
			$\bm r = \bm b-A\bm x  $\\
			\While{$\|r\|_2 <$ tol \textbf{and} $i <$ maxIter} {
				$\bm x  \leftarrow \bm x + \omega \bm r  $\\
				$\bm r  \leftarrow \bm b - A\bm x$
			}
			\Return $\bm x$;
			\caption{Optimal Richardson Iteration}
		}
	\end{algorithm}

	\begin{algorithm}[H]
		\Fn(){\FuncSty{JacobiIteration(}{$A$,$\bm b$,$\bm x_0$,tol,maxIter }\FuncSty{)}}{
			\KwIn{\\$A$:The matrix to find the solution to \\  $\bm b$: The resultant vector in $A\bm x = \bm b$  \\ $\bm x_0$: The initial guess \\ maxIter: The maximum of iterations}
			\KwOut{x: The solution to $A\bm x = \bm b$}
			\SetAlgoLined
			\SetNoFillComment
			\vspace{3mm}
			$M \leftarrow $diag($A$) \\
			$N\leftarrow  A- M$\\
			$\bm x \leftarrow  \bm x_0$\\
			$\bm r \leftarrow \bm b-A\bm x  $\\
			\While{$\|r\|_2 <$ tol \textbf{and} $i <$ maxIter} {
				$\bm x  \leftarrow M^{-1}(\bm x + \bm b - N\bm x)  $\\
				$\bm r  \leftarrow \bm b - A\bm x$ \\
				$i = i+1$\\
			}
			\Return $\bm x$
			\caption{Jacobi Iteration}
		}
	\end{algorithm}

	\begin{algorithm}[H]
		\Fn(){\FuncSty{GaussSiedelIteration(}{A,$\bm b$,$\bm x_0$,tol,maxIter }\FuncSty{)}}{
			\KwIn{\\A: The matrix to find the solution to \\  $\bm b$: The resultant vector in $A\bm x = \bm b$  \\ $\bm x_0$: The initial guess \\  maxIter: The maximum of iterations}
			\KwOut{x: The solution to $A\bm x = \bm b$}
			\SetAlgoLined
			\SetNoFillComment
			\vspace{3mm}
			$M \leftarrow $diag($A$)+lower($A$) \\
			$N\leftarrow  A- M$\\
			$\bm x \leftarrow  \bm x_0$\\
			$\bm r \leftarrow \bm b-A\bm x  $\\
			$i \leftarrow 0$ \\
			\While{$\|r\|_2 <$ tol \textbf{and} $i <$ maxIter}  {
				$\bm x  \leftarrow M^{-1}(\bm x + \bm b - N\bm x)  $\\
				$\bm r  \leftarrow \bm b - A\bm x$ \\
				$i = i+1$
			}
			\Return $\bm x$;
			\caption{Gauss-Sidel Iteration}
		}
	\end{algorithm}
\end{solution}
\qs{Steepest Descent and Conjugate Gradient}
{
	\begin{enumerate}[label=(\alph*)]
		\item Let $A$ be a S.P.D. matrix. Show that $(\boldsymbol{x}, \boldsymbol{y})_A:=\boldsymbol{x}^T A \boldsymbol{y}$ for $\boldsymbol{x}, \boldsymbol{y} \in \mathbb{R}^n$ forms an inner product on $\mathbb{R}^n$.
		\item Using part (a), conclude that $\|\boldsymbol{x}\|=(\boldsymbol{x}, \boldsymbol{x})_A^{1 / 2}$ for $\boldsymbol{x} \in \mathbb{R}^n$ is a norm on $\mathbb{R}^n$.

		      Hint: You can assume the Cauchy-Schwarz inequality $\left|(\boldsymbol{x}, \boldsymbol{y})_A\right| \leq\|\boldsymbol{x}\|_A\|\boldsymbol{y}\|_A$ holds.
		\item For the method of Steepest Descent, show that $\nabla f\left(\boldsymbol{x}_k\right)$ and $\nabla f\left(\boldsymbol{x}_{k+1}\right)$ are orthogonal (i.e. zig-zaging behavior), where $f(\boldsymbol{y})=\frac{1}{2} \boldsymbol{y}^T A \boldsymbol{y}-\boldsymbol{y}^T \boldsymbol{b}$. Hint: Recall how the step size for Steepest Descent is determined.
		\item Repeat the experiment from $\mathbf{Q 3}(\mathbf{b})$ with $\boldsymbol{b}=(1, \ldots, 1)^T \in \mathbb{R}^{1000}$ and $\boldsymbol{x}_0=(1,0, \ldots, 0)^T \in \mathbb{R}^{1000}$ using the method of Steepest Descent and Conjugate Gradient. Generate a plot comparing the $\log$ of their residual in $\ell_2$ norm versus iterations up to 5000 . Rank their performance by comparing the iterations needed to reach the residual tolerance of $10^{-14}$, as well as versus the classical iterative methods. Verify your CG method terminates after the desired number of iterations. Use sparse representation when appropriate.
	\end{enumerate}
}
\begin{solution}
	\begin{enumerate}[label=(\alph*)]
		\item $\left( .,. \right) $ is an inner-product if :
		      \begin{enumerate}[label=(\roman*)]
			      \item  Conjugate Symmetery: \\
			            \[
				            \left( x,y \right)_{A}  = {\left(y,x\right)_A}
				            .\]
			      \item Linearity \\
			            \[
				            \left( a\vec{x}+b\vec{y}, \vec z \right)_A  = a \left( \vec x,\vec z  \right)_A +b\left(\vec  y,\vec z \right)_A
				            .\]
			      \item Positive-Definiteness:\\
			            \[
				            \left( \vec x, \vec x \right)_A  > 0
				            .\]
		      \end{enumerate}
		      \begin{enumerate}[label = (\roman*)]
			      \item \[
				            \left( x,y \right)_A = x^{T}Ay = y^{T}Ax = \left( y,x \right)_A
				            .\]
			      \item
			            \[
				            \left( a\vec{x}+b\vec{y}, \vec z \right)_A = (a\vec{x}+b\vec{y})^{T} A \vec z=  a\vec{x}^{T} A \vec z+b\vec{y}^{T} A \vec z  = a \left( \vec x,\vec z  \right)_A +b\left(\vec  y,\vec z \right)_A
				            .\]
			      \item
			            \[
				            \left( x,x \right)_A = \vec x^{T} A  \vec x >0 \quad \text{As $A$ as is SPD}
				            .\]
			            Therefore, $\left( .,. \right)_A $ is an inner-product.
		      \end{enumerate}
		\item $\|.\|_A$ is an norm if :
		      \begin{enumerate}[label=(\roman*)]
			      \item  Positive Definitness : \\
			            \[
				            \|x\|_A > 0 \quad \forall \vec x \neq 0 \quad \land \quad \|\vec x\|_A = 0 \iff \vec x = \vec 0
				            .\]
			      \item Scalar Multiplication\\
			            \[
				            \|\lambda \vec x \|_A = \lambda \|\vec x\|_A
				            .\]
			      \item Sub-additivity (Triangle Inequality):\\
			            \[
				            \|\vec x+ \vec y\|_A = \| \vec x\|_A  + \| \vec y \|_A
				            .\]

		      \end{enumerate}

		      \begin{enumerate}[label=(\roman*)]
			      \item  	Let $x\in \RR^{n} $ and $\vec x \neq \vec 0$    \[
				            \|x\|_A = \sqrt{\vec x^{T}A\vec x}  >0  \quad \text{,as $A$ is SPD}
				            .\]
			            Let $\|\vec x \|_A =0$
			            \[
				            \|\vec x \|_A =0 = \sqrt{\vec x^{T} A \vec x}  \iff \vec x = 0 \quad \text{, as $A$ is SPD}
				            .\]

			      \item Scalar Multiplication\\
			            \[
				            \|\lambda \vec x \|_A = \sqrt{\lambda \vec x^{T} A \lambda \vec x} = \sqrt{\lambda^{2} \vec x^{T} A \vec x}    = \lambda \|\vec x\|_A
				            .\]
			      \item Sub-additivity (Triangle Inequality):\\
			            \begin{align*}
				            \|\vec x+ \vec y\|_A
				             & = \sqrt{\left( \vec x+ \vec y \right)^{T} A \left( \vec x+ \vec y \right)  } \\
				             & =  \sqrt{\vec x^{T}A\vec x + \vec x^{T}A\vec y + \vec y^{T}A\vec x +\vec y^{T}A\vec y  } \\
				             & = \sqrt{\vec x^{T}A\vec x + 2\vec x^{T}A\vec y +\vec y^{T}A\vec y  } \\
				             & =   \sqrt{\|\vec x\|_A^2+ 2 \left( \vec x,\vec y \right)  +\|\vec y\|_A^2 } \\
				             & \le  \sqrt{\|\vec x\|_A^{2}+ 2 \|\vec x\|_A\|\vec y\|_A   +\|\vec y\|_A^2 } \\
				             & =  \| \vec x\|_A  + \| \vec y \|_A
				            .\end{align*}
			            Therefore, $\|x\|_A$ is valid norm.
		      \end{enumerate}
	\end{enumerate}
\end{solution}
\end{document}
